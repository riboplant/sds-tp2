\documentclass[12pt,a4paper]{article}
\usepackage[utf8]{inputenc}
\usepackage{amsmath, amssymb}
\usepackage{graphicx}
\usepackage{caption}
\usepackage{geometry}
\usepackage{bm} % Para vectores en negrita
\usepackage{siunitx} % Para unidades
\usepackage{cite}

\geometry{margin=2.5cm}

\title{\textbf{Simulación del Modelo de Vicsek Off-Lattice}}
\author{Nombre Apellido \\ Universidad \\ Materia: Sistemas de Simulación}
\date{\today}

\begin{document}

\maketitle

\section{Introducción}
En este trabajo se estudia el \textit{modelo de Vicsek off-lattice} \cite{vicsek1995novel}, un autómata celular continuo que describe el comportamiento colectivo de partículas autopropulsadas en dos dimensiones.  
El objetivo es analizar la transición entre estados de desorden y orden colectivo en función de la densidad de partículas y la amplitud de ruido.

\section{Modelo}
El modelo considera $N$ partículas puntuales distribuidas en un dominio cuadrado de lado $L$ con condiciones periódicas.  
Cada partícula $i$ posee una posición $\bm{x}_i(t)$ y una dirección $\theta_i(t)$ con módulo de velocidad fijo $v$.  

En cada paso temporal:
\begin{enumerate}
    \item Se identifican los vecinos dentro de un radio $r$ (incluyendo la propia partícula).
    \item Se calcula la dirección promedio:
    \begin{equation}
        \langle \theta(t) \rangle_r = \arctan2\left( \langle \sin(\theta(t)) \rangle_r, \langle \cos(\theta(t)) \rangle_r \right)
        \label{eq:promedio}
    \end{equation}
    \item Se agrega ruido uniforme $\Delta\theta \in [-\eta/2, \eta/2]$:
    \begin{equation}
        \theta_i(t+1) = \langle \theta(t) \rangle_r + \Delta\theta
        \label{eq:ruido}
    \end{equation}
    \item Se actualiza la posición:
    \begin{equation}
        \bm{x}_i(t+1) = \bm{x}_i(t) + v \cdot \begin{bmatrix} \cos(\theta_i(t+1)) \\ \sin(\theta_i(t+1)) \end{bmatrix}
        \label{eq:posicion}
    \end{equation}
\end{enumerate}

El grado de alineamiento se mide con el \textbf{parámetro de orden}:
\begin{equation}
    v_a = \frac{1}{N v} \left| \sum_{i=1}^N \bm{v}_i \right|
    \label{eq:va}
\end{equation}
donde $v_a \approx 1$ indica orden total y $v_a \approx 0$ desorden.

\section{Implementación}
El modelo fue implementado en \texttt{Python}, empleando \texttt{numpy} para operaciones vectorizadas y \texttt{matplotlib} para visualización.  
Se utilizó un algoritmo de búsqueda de vecinos de complejidad $O(N^2)$, suficiente para $N \lesssim 10^3$.

\section{Metodología}
Para obtener resultados estadísticamente confiables:
\begin{itemize}
    \item Se definieron parámetros iniciales $(N, L, r, v, \eta)$ y un número de iteraciones $T$.
    \item Se descartaron las primeras $T_{\mathrm{trans}}$ iteraciones como \textit{transitorio}, de manera de medir solo el régimen estacionario.
    \item Para cada conjunto de parámetros, se ejecutaron $R$ \textbf{realizaciones independientes} con condiciones iniciales aleatorias (posiciones y direcciones).
    \item En cada realización se calculó el parámetro de orden $v_a(t)$ y se promedió sobre el régimen estacionario:
    \begin{equation}
        \langle v_a \rangle = \frac{1}{T_{\mathrm{med}}} \sum_{t=T_{\mathrm{trans}}+1}^{T} v_a(t)
        \label{eq:promedio_va}
    \end{equation}
    donde $T_{\mathrm{med}} = T - T_{\mathrm{trans}}$.
    \item El valor final reportado en gráficos de $\langle v_a \rangle$ vs. $\eta$ corresponde al promedio sobre $R$ realizaciones, y se incluyen barras de error calculadas como el desvío estándar.
\end{itemize}

\section{Simulaciones}
Las simulaciones se realizaron variando la amplitud de ruido $\eta$ y, opcionalmente, la densidad $\rho = N/L^2$.  
Cada configuración se ejecutó durante $T$ iteraciones, descartando las primeras como \textit{transitorio}, y se promediaron los resultados sobre $R$ realizaciones independientes para reducir fluctuaciones estadísticas.

\section{Resultados y Conclusiones}
En la Fig.~\ref{fig:va_vs_t} se observa la evolución temporal del parámetro de orden para $\eta = 0.1$, $N = 300$, $L = \SI{7}{m}$, $v = \SI{0.03}{m/s}$ y $r = \SI{1}{m}$.  
El sistema alcanza rápidamente un estado de orden colectivo ($v_a \to 1$) debido al bajo ruido y alta densidad de vecinos.

\begin{figure}[h]
    \centering
    \includegraphics[width=0.7\textwidth]{va_vs_t.png}
    \caption{Evolución temporal del parámetro de orden $v_a$ según la Ec.~\eqref{eq:va}, para una única realización.}
    \label{fig:va_vs_t}
\end{figure}

La Fig.~\ref{fig:va_vs_eta} muestra el valor promedio $\langle v_a \rangle$ en función de la amplitud de ruido $\eta$, promediado sobre $R$ realizaciones. Se aprecia una transición de fase de segundo orden entre estados ordenados y desordenados, con un valor crítico $\eta_c$ intermedio.

\begin{figure}[h]
    \centering
    \includegraphics[width=0.7\textwidth]{va_vs_eta.png}
    \caption{Parámetro de orden promedio $\langle v_a \rangle$ vs. ruido $\eta$, promediado sobre $R$ realizaciones. Barras de error: desvío estándar.}
    \label{fig:va_vs_eta}
\end{figure}

En conclusión:
\begin{itemize}
    \item Para bajos valores de $\eta$, el sistema tiende rápidamente al orden colectivo global.
    \item A medida que $\eta$ aumenta, se observa una pérdida gradual de alineamiento, hasta llegar a desorden completo.
    \item La densidad $\rho$ influye en la posición de la transición, con mayores densidades favoreciendo el orden.
\end{itemize}

\section*{Referencias}
\begin{thebibliography}{9}
\bibitem{vicsek1995novel}
T.~Vicsek, A.~Czirok, E.~Ben-Jacob, I.~Cohen, and O.~Shochet, ``Novel type of phase transition in a system of self-driven particles,'' \textit{Physical Review Letters}, vol.~75, no.~6, pp. 1226--1229, 1995.
\end{thebibliography}

\end{document}
